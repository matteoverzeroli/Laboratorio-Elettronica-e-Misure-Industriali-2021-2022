\chapter{Laboratorio 3}
\section{Introduzione}
In questa esperienza di laboratorio è stato analizzato il circuito di amplificazione \textit{common emitter amplifier} con degenerazione di emettitore, nella versione con alimentazione doppia (positiva e negativa) e nella versione single-ended (\Fig\ref{fig:commonemitter}).
\begin{figure}[h!]
	\centering
	A
	\includegraphics[width=0.4\linewidth]{./OtherFiles/Laboratorio 3/common emitter}
	B
	\includegraphics[width=0.4\linewidth]{./OtherFiles/Laboratorio 3/common emitter_se}
	\caption{Schematico del circuito common emitter con degenerazione di emettitore nella versione con alimentazione doppia (A) e singola (B).}
	\label{fig:commonemitter}
\end{figure}
Come verrà discusso in seguito, la resistenza di emettitore permette di ottenere un guadagno che è indipendente dalla transconduttanza del transistor (più precisamente, permette di ridurre in modo significativo la dipendenza del guadagno dalla transconduttanza). \`E possibile realizzare anche una versione dell'amplificatore a emettitore comune senza la resistenza di degenerazione R\sub{E}. Tuttavia, il guadagno risulterebbe pari a $\frac{v_o}{v_i}=-g_m R_c$, dove $g_m=\frac{I_C}{\phi_T}$ e $\phi_T=\frac{k T}{q}$, introducendo una dipendenza dalla temperatura T a cui il circuito opera.

\section{Common emitter amplifier: punto di lavoro}
Prima di procedere alla realizzazione dei circuito analizziamone il punto di lavoro. 

\noindent
Consideriamo prima la versione con alimentazione doppia. Supponiamo il transistor in regione attiva diretta, con corrente di base I\sub{B} nulla. Di conseguenza, la tensione al nodo V\sub{B} sarà nulla (corrente nella resistenza R\sub{B} nulla). Ricordando che $V_{BE}=\SI{-0.7}{\volt}$, si deriva che $V_E=\SI{-0.7}{\volt}$. Per cui, la corrente che attraversa la resistenza R\sub{E} è pari a (legge di Ohm) $I_E=\frac{V_E-(\SI{-2}{\volt})}{R_E}$. Dal bilancio di correnti nel transistor, si ricava che $I_C+I_B=I_E$, da cui $I_C=I_E$. La tensione al nodo V\sub{o} si ricava tramite legge di Ohm: $I_C=\frac{\SI{10}{\volt}-V_o}{R_C}$, da cui $V_o= \SI{10}{\volt}-R_CI_C$. Inoltre, ricordiamo che $g_m=\frac{I_C}{\phi_T}$, con $I_C$ corrente di collettore stazionaria e $\phi_T\simeq\SI{26}{\milli\volt}$. Bisognerà verificare che V\sub{BE} sia positiva, in modo da confermare l'ipotesi che il transistor si trova in regione attiva diretta.
\begin{figure}[h!]
	\centering
	\includegraphics[width=0.4\linewidth]{./OtherFiles/Laboratorio 3/common emitter-punto di lavoro-printout}
	\caption{Analisi del punto di lavoro del circuito common emitter amplifier con alimentazione positiva e negativa.}
	\label{fig:commonemitter_DC}
\end{figure}

\noindent
Sostituendo i valori dei componenti utilizzati nelle formule ricavate, otteniamo:
\begin{table}[h!]
	\centering
	\begin{tabular}{c|c|c|c|c|c}
		\hline
		V\sub{B} [V] & V\sub{E} [V] & V\sub{O} [V] & I\sub{B} [A] & I\sub{E} [mA] & I\sub{C} [mA] \\ \hline
		0 & -0.7 & 7.183  & 0 & 0.7222 & 0.7222 \\ \hline
	\end{tabular}
\end{table}

\noindent
L'analisi del punto di lavoro del circuito \textit{common emitter amplifier} nella versione con alimentazione singola è simile a quella appena ricavata (\Fig\ref{fig:commonemitter_se_DC}).
\begin{figure}[h!]
	\centering
	\includegraphics[width=0.4\linewidth]{./OtherFiles/Laboratorio 3/common emitter_se-punto di lavoro-printout}
	\caption{Analisi del punto di lavoro del circuito common emitter amplifier con alimentazione singola.}
	\label{fig:commonemitter_se_DC}
\end{figure}
Supponendo che la corrente di base sia zero (ossia che $\beta\to\infty$), da un bilancio di corrente al nodi V\sub{B} e nel transistor, otteniamo che $I_1=I_2$ e $I_C=I_E$. Si noti che il condensatore è stato considerato come un circuito aperto nell'analisi DC. Infatti, il suo ruolo è quello di disaccoppiare in continua i nodi V\sub{B} e V\sub{i}, in modo che la tensione di offset imposta dal partitore di tensione formato da R\sub{1} e R\sub{2} non sia portata a massa dal generatore V\sub{i}. Infatti, come già discusso in precedenza nell'analisi dell'emitter follower single-ended, le resistenze R\sub{1} e R\sub{2} hanno il compito di alzare la tensione nel nodo V\sub{B}, in modo da mantenere il transistor in zona attiva diretta.

\noindent
Ricaviamo quindi la tensione V\sub{B} utilizzando la legge di Ohm:
\begin{equation}
	\begin{split}
		I=I_1&=I_2 \\
		\frac{\SI{10}{\volt}-V_B}{R_1}&=\frac{V_B-\SI{0}{\volt}}{R_2} \\
		V_B&=\frac{R_2}{R_1+R_2}\SI{10}{\volt}
	\end{split}
\end{equation}
La tensione al nodo V\sub{E} si ricava dall'equazione $V_{BE}=V_B-V_E=\SI{0.7}{\volt}$ (transistor in zona attiva diretta), da cui $V_E=V_B-\SI{0.7}{\volt}$. Conoscendo V\sub{E} è ora possibile calcolare la corrente I\sub{E} e I\sub{C} tramite la legge di Ohm: $I_E=I_C=\frac{V_E-\SI{0}{\volt}}{R_E}$. 
Infine, la tensione nel nodo V\sub{O} sarà $V_O=\SI{10}{\volt}-I_CR_C$. Si dovrà inoltre verificare che la tensione V\sub{CB} sia maggiore di zero.

\noindent
Sostituendo i valori dei componenti utilizzati nel circuito otteniamo i seguenti valori, che dovranno essere confrontati con quelli misurati sul circuito reale:
\begin{table}[h!]
	\centering
	\begin{tabular}{c|c|c|c|c|c}
		\hline
		V\sub{B} [mV] & V\sub{E} [mV] & V\sub{O} [V] & I\sub{B} [A] & I\sub{E} [mA] & I\sub{C} [mA]\\ \hline
		800 & 100 & 9.783 & 0 & 0.055 & 0.055\\ \hline
	\end{tabular}
\end{table}

\section{Common emitter amplifier: analisi per piccolo segnale}\label{sec:33}
Si procede ora con l'analisi di piccolo segnale, utilizzando il modello per piccolo segnale a bassa frequenza del transistor e spegnendo i generatori di grandezze continue.

\noindent
Considerando il circuito a emettitore comune con alimentazione doppia si ottiene il circuito equivalente in figura \ref{fig:commonemitter_AC}.
\begin{figure}[h!]
	\centering
	\includegraphics[width=0.4\linewidth]{./OtherFiles/Laboratorio 3/common emitter-piccolo segnale-printout}
	\caption{Analisi per piccolo segnale del circuito common emitter amplifier con alimentazione positiva e negativa.}
	\label{fig:commonemitter_AC}
\end{figure}
Dal momento che nella resistenza R\sub{B} non passa corrente, $v_b=v_i$. Inoltre si possono ricavare le seguenti equazioni (utilizzando legge di Ohm e bilanci di corrente):
\begin{equation}
	\begin{cases}
		i_c=\frac{-v_o}{R_C} \\
		i_c=gmv_{be}=gmv_{ie} \\
		i_c=\frac{v_e}{R_E} \\
		i_e=i_c
	\end{cases}
\end{equation}
Risolvendo il sistema, è possibile calcolare la funzione di trasferimento del circuito:
\begin{equation}
	\frac{v_o}{v_i}=-\frac{R_Cg_m}{1+R_Eg_m}\;\underset{g_mR_E\gg 1}{\simeq}\;-\frac{R_C}{R_E}.
\end{equation}
Il guadagno del circuito è quindi determinato dal rapporto tra la resistenza di collettore e quella di emettitore. Inoltre, la presenza di un segno meno nella funzione di trasferimento indica che il circuito introduce uno sfasamento di \SI{-180}{\degree} tra il segnale in ingresso e quello in uscita (amplificatore invertente). 

\noindent
L'analisi di piccolo segnale nella versione con singola alimentazione è identica a quella presentata precedentemente sotto l'ipotesi che $C\to\infty$, ossia che il condensatore si comporti come un corto circuito per frequenze del segnale sufficientemente alte. Sotto questa ipotesi semplificativa, $v_b=v_i$. Con la risoluzione del medesimo sistema di equazioni si determina che il guadagno è $\frac{v_o}{v_i}\simeq-\frac{R_C}{R_E}.$
\begin{figure}[h!]
	\centering
	\includegraphics[width=0.4\linewidth]{./OtherFiles/Laboratorio 3/common emitter_se-piccolo segnale-printout}
	\caption{Analisi per piccolo segnale del circuito common emitter amplifier con alimentazione singola.}
	\label{fig:commonemitter_se_AC}
\end{figure}

\noindent
Sostituendo i valori delle resistenze da utilizzare nei nostri circuiti ci aspettiamo un guadagno di:
\begin{equation}
	\frac{v_o}{v_i}=-\frac{\SI{3.9}{\kilo\ohm}}{\SI{1.8}{\kilo\ohm}}\simeq -2.17.
\end{equation}

\section{Componenti e misure}
I due circuiti sono stati realizzati in laboratorio (\Fig\ref{fig:commonemitter_circuito}).
\begin{figure}[h!]
	\centering
	A)
	
	\includegraphics[width=0.57\linewidth]{./ImageFiles/Laboratorio 3/IMG\_20220524\_110713_2}
	
	B)
	
	\includegraphics[width=0.57\linewidth]{./ImageFiles/Laboratorio 3/IMG\_20220524\_122237\_2}
	\caption{Foto del circuito realizzato in laboratorio nella versione con doppia (A) e singola (B) alimentazione.}
	\label{fig:commonemitter_circuito}
\end{figure}

\noindent
Per la versione con alimentazione doppia sono stati utilizzati i seguenti componenti:
\begin{itemize}
	\item transistor 2N3904;
	\item una resistenza da \SI{3.9}{\kilo\ohm} per realizzare la resistenza R\sub{C};
	\item una resistenza da \SI{1.8}{\kilo\ohm} per realizzare la resistenza R\sub{E};
	\item una resistenza da \SI{270}{\ohm} per realizzare la resistenza R\sub{B}.
\end{itemize}
Invece per la versione single-ended sono stati utilizzati:
\begin{itemize}
	\item transistor 2N3904;
	\item una resistenza da R\sub{11}=\SI{4.6}{\kilo\ohm} in serie a due resistenze da R\sub{12}=\SI{220}{\kilo\ohm} e R\sub{13}=\SI{220}{\kilo\ohm} connesse in parallelo per realizzare la resistenza R\sub{1};
	\item una resistenza da \SI{12}{\kilo\ohm} per realizzare la resistenza R\sub{2}.
	\item una resistenza da \SI{1.8}{\kilo\ohm} per realizzare la resistenza R\sub{E};
	\item una resistenza da \SI{3.9}{\kilo\ohm} per realizzare la resistenza R\sub{C};
\end{itemize}

Inoltre, sono stati utilizzati i seguenti strumenti:
\begin{itemize}
	\item alimentatore da banco con tensione positiva \SI{10}{\volt}, tensione negativa \SI{-2}{\volt} (utilizzata solo per il circuito con doppia alimentazione)e limite in corrente di \SI{50}{\milli\ampere};
	\item oscilloscopio a due canali;
	\item generatore di forme d'onda;
	\item multimetro da banco.
\end{itemize}

\noindent
Nelle seguenti tabelle vengono riportate le misure dei componenti usati effettuate tramite il multimetro:

\begin{table}[h!]
	\centering
	\begin{tabular}{c|c|c}
		\hline
		Componente & Valore nominale & Valore misurato \\ \hline
		R\sub{C} & \SI{3.9}{\kilo\ohm} & \SI{3.962}{\kilo\ohm} \\ \hline
		R\sub{E} & \SI{1.8}{\kilo\ohm} & \SI{1.825}{\kilo\ohm} \\ \hline
		R\sub{B} & \SI{270}{\kilo\ohm} & \SI{274.6}{\ohm} \\ \hline
		Vd\sub{B-E} & $\simeq$ \SI{0.7}{\volt} & \SI{0.700}{\volt} \\ \hline
		Vd\sub{B-C} & $\simeq$ \SI{0.7}{\volt} & \SI{0.679}{\volt} \\ \hline
	\end{tabular}
	\caption{Valori dei componenti utilizzati per il circuito common emitter amplifier con alimentazione doppia misurati con il multimetro.}
\end{table}
\begin{table}[h!]
	\centering
	\begin{tabular}{c|c|c}
		\hline
		Componente & Valore nominale & Valore misurato \\ \hline
		R\sub{11} &\SI{220}{\kilo\ohm} & \SI{222.640}{\kilo\ohm} \\ \hline
		R\sub{12} &\SI{220}{\kilo\ohm} & \SI{221.290}{\kilo\ohm} \\ \hline
		R\sub{13} &\SI{4.6}{\kilo\ohm} & \SI{4.639}{\kilo\ohm} \\ \hline
		R\sub{2} &\SI{12}{\kilo\ohm} & \SI{11.883}{\kilo\ohm} \\ \hline
		R\sub{C} & \SI{3.9}{\kilo\ohm} & \SI{3.962}{\kilo\ohm} \\ \hline
		R\sub{E} & \SI{1.8}{\kilo\ohm} & \SI{1.825}{\kilo\ohm} \\ \hline
		Vd\sub{B-E} & $\simeq$ \SI{0.7}{\volt} & \SI{0.700}{\volt} \\ \hline
		Vd\sub{B-C} & $\simeq$ \SI{0.7}{\volt} & \SI{0.679}{\volt} \\ \hline
	\end{tabular}
	\caption{Valori componenti utilizzati per il circuito common emitter amplifier con alimentazione singola misurati con il multimetro.}
\end{table}

\noindent
Una volta realizzato il circuito, sono state misurate le tensioni ai nodi V\sub{B}, V\sub{O} e V\sub{E}, con il nodo V\sub{i} collegato a massa, da cui è possibile ricavare i valori delle correnti I\sub{C}, I\sub{E} e I\sub{B}. Di seguito sono riportati i risultati per la versione con alimentazione doppia e singola:
\begin{table}[h!]
	\centering
	\begin{tabular}{c|c|c|c|c|c}
		\hline
		V\sub{B} [mV] & V\sub{E} [V] & V\sub{O} [V] & I\sub{B} [mA] & I\sub{E} [mA] & I\sub{C} [mA]\\ \hline
		-1.056 & -0.655 & 7.080 & -0.0001 & 0.7369 & 0.7370 \\ \hline
	\end{tabular}
	\caption{Misure dei valori stazionari di tensione e corrente nel circuito common emitter amplifier con alimentazione doppia.}
\end{table}
\begin{table}[h!]
	\centering
	\begin{tabular}{c|c|c|c|c|c}
		\hline
		V\sub{B} [mV] & V\sub{E} [mV] & V\sub{O} [V] & I\sub{B} [mA] & I\sub{E} [mA] & I\sub{C} [mA]\\ \hline
		920 & 304 & 9.334 & 0.0150 & 0.1666 & 0.1681 \\ \hline
	\end{tabular}
	\caption{Misure dei valori stazionari di tensione e corrente nel circuito common emitter amplifier con alimentazione singola.}
\end{table}

\noindent
\`E importante osservare che la corrente di base I\sub{B} misurata nel circuito con alimentazione doppia è negativa. Infatti, la corrente di emettitore risulta essere minore della corrente di collettore. Tuttavia, ciò non è ammissibile per costruzione del transistor. Questo risultato però non deve sorprendere in quanto l'errore è dell'ordine di $10^{-7}$ A. Trattandosi di una corrente molto piccola, è difficile misurarla con gli strumenti a disposizione.

In seguito è stata misurata la risposta del circuito a un ingresso sinusoidale. 

\noindent
Per il circuito con alimentazione doppia è stato utilizzato un ingresso sinusoidale di frequenza \SI{1}{\kilo\hertz} e ampiezza picco-picco di \SI{1}{\volt} (\Fig\ref{fig:commonemitter_guadagno}). Calcolando il rapporto tra la resistenza di collettore e quella di emettitore (che determinano il guadagno del circuito) il guadagno atteso è di circa 2.17, mentre il guadagno misurato è di circa 2.14. Si noti inoltre che i segnali sono in controfase, così come è stato ricavato nell'analisi di piccolo segnale (segno meno nell'espressione del guadagno). 

\begin{figure}[h!]
	\centering
	\includegraphics[width=0.7\linewidth]{./ImageFiles/Laboratorio 3/TEK00006}
	\vspace{1cm}
	
	\includegraphics[width=0.7\linewidth]{./ImageFiles/Laboratorio 3/TEK00007}
	\caption{Confronto tra il segnale in ingresso (CH1) e in uscita (CH2) al circuito common emitter amplifier con alimentazione doppia. L'onda sinusoidale in ingresso ha ampiezza picco-picco di \SI{1}{\volt} e frequenza di \SI{1}{\kilo\hertz}.}
	\label{fig:commonemitter_guadagno}
\end{figure}

\noindent
Risultati analoghi sono stati ottenuti sul circuito nella versione single-ended, dove in ingresso è stato applicato un segnale sinusoidale prima con ampiezza picco-picco di \SI{100}{\milli\volt} e poi di \SI{1}{\volt}, mantenendo costante la frequenza a \SI{1}{\kilo\hertz} (\Fig\ref{fig:commonemitter_se_guadagno}). Il guadagno atteso è il medesimo del circuito precedente (rapporto tra resistenza di collettore e emettitore). In questo caso, il guadagno reale ottenuto è leggermente inferiore e pari a 1.96.

\noindent
\`E importante analizzare il comportamento del circuito nel caso in cui in ingresso è stata fornita un'onda sinusoidale di ampiezza picco-picco di \SI{1}{\volt} (\Fig\ref{fig:commonemitter_se_guadagno}(B)). Infatti, si nota che l'uscita satura. Ciò accade poiché al nodo V\sub{O}, che si trova ad una tensione di \SI{9.334}{\volt}, si somma una componente di segnale di ampiezza \SI{1}{\volt}, che porterebbe l'uscita ad una tensione superiore all'alimentazione positiva di \SI{10}{\volt}. Ma ciò non è possibile. 

\begin{figure}[h!]
	\centering
	(A)
	
	\includegraphics[width=0.7\linewidth]{./ImageFiles/Laboratorio 3/TEK00010}
	\vspace{1cm}
	
	(B)
	
	\includegraphics[width=0.7\linewidth]{./ImageFiles/Laboratorio 3/TEK00011}
	\caption{Confronto tra il segnale in ingresso (CH1) e in uscita (CH2) al circuito common emitter amplifier con alimentazione singola. L'onda sinusoidale in ingresso ha ampiezza picco-picco di \SI{100}{\milli\volt} (A) e \SI{1}{\volt} (B) e frequenza di \SI{1}{\kilo\hertz}.}
	\label{fig:commonemitter_se_guadagno}
\end{figure}

\noindent
Nella successiva esperienza di laboratorio si analizzerà il comportamento del circuito con un segnale sinusoidale di frequenza minore, dove gli effetti della capacità di disaccoppiamento non sono trascurabili.